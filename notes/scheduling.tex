\documentclass[12pt]{article}
\usepackage{amssymb, amsmath, amsthm, amsfonts}
\usepackage[alphabetic]{amsrefs}
\usepackage{mathrsfs,comment}
\usepackage{float}
\usepackage[all,arc,2cell]{xy}
\usepackage{enumerate}
\usepackage{tcolorbox}
\usepackage[margin=0.75in]{geometry}
\usepackage{MnSymbol}
\usepackage{pdfpages}
\usepackage{tikz}
\usetikzlibrary{arrows,automata,calc,positioning}
\usepackage[colorlinks = true,
linkcolor = blue,
urlcolor  = blue,
citecolor = blue,
anchorcolor = blue]{hyperref}

\usepackage{cleveref}


%theoremstyle{plain} --- default
\newtheorem{theorem}{Theorem}[section]
\newtheorem{corollary}[theorem]{Corollary}
\newtheorem{conjecture}[theorem]{Conjecture}
\newtheorem{proposition}[theorem]{Proposition}
\newtheorem{lemma}[theorem]{Lemma}

\theoremstyle{definition}
\newtheorem{definition}[theorem]{Definition}
\newtheorem{example}[theorem]{Example}
\newtheorem{remark}[theorem]{Remark}




\makeatother
\numberwithin{equation}{section}



%%%Commands and shortcuts%%%%
%

\newcommand{\R}{\mathbb{R}}
\newcommand{\Q}{\mathbb{Q}}
\newcommand{\Z}{\mathbb{Z}}
\newcommand{\N}{\mathbb{N}}
\newcommand{\xb}{\mathbf{x}}
\newcommand{\yb}{\mathbf{y}}
\newcommand{\ab}{\mathbf{a}}
\newcommand{\zb}{\mathbf{z}}
\newcommand{\func}[3]{#1\colon #2 \rightarrow #3}
\newcommand{\sumn}{\sum_{i=1}^{n}}
\newcommand{\innerp}[2]{\langle #1,#2 \rangle}
\newcommand{\dsq}{d_{sq}}
\newcommand{\dtc}{d_{tc}}
\newcommand{\ifff}{if and only if }
\newcommand{\st}{such that }
\newcommand{\te}{there exists }
\newcommand{\wrt}{with respect to }
\newcommand{\wloG}{without loss of generality }
\newcommand{\Wlog}{Without loss of generality }
\newcommand{\ssteq}{\subseteq}
\newcommand{\setst}{\hspace{1mm} | \hspace{1mm} }
\newcommand{\Bs}{\mathscr{B}}
\newcommand{\Us}{\mathscr{U}}
\newcommand{\T}{\mathcal{T}}
\newcommand{\Y}{\mathcal{Y}}
\newcommand{\B}{\mathcal{B}}
\newcommand{\G}{\mathcal{G}}
\newcommand{\Pc}{\mathcal{P}}
\newcommand{\Sc}{\mathcal{S}}
\newcommand{\C}{\mathcal{C}}
\newcommand{\A}{\mathcal{A}}
\newcommand{\Lc}{\mathcal{L}}
\newcommand{\onen}{\{1,\ldots,n\}}
\newcommand{\finv}[2]{#1^{-1}\left(#2\right)}		
\newcommand{\es}{\varnothing}	
\newcommand{\ol}[1]{\overline{#1}}	
\newcommand{\xnn}{(\mathbf{x}_n)_{n \in \N}}
\newcommand{\sft}{\mathbf{SFT_1}}		
\newcommand{\cop}{\sqcup}	
\newcommand{\cyc}[1]{#1_1\ldots #1_n#1_1}	
\newcommand{\pcyc}[2]{#2(#1_1)\ldots #2(#1_n)#2(#1_1)}	
\newcommand{\cyci}[2]{#1_1\ldots #1_{#2}#1_1}
\newcommand{\pcyci}[3]{#3(#1_1)\ldots #3(#1_{#2})#3(#1_1)}
\newcommand{\xh}{\bar{x}}	
\newcommand{\yh}{\bar{y}}		
\newcommand{\equivcls}[1]{#1/{\sim}}																																													

\DeclareMathOperator{\Cl}{Cl}
\DeclareMathOperator{\Bd}{Bd}
\DeclareMathOperator{\Int}{Int}
\DeclareMathOperator{\id}{id}

\renewcommand{\epsilon}{\varepsilon}
\renewcommand{\phi}{\varphi}
\renewcommand\qedsymbol{$\blacksquare$}

\title{Scheduling}
\author{Luke Meszar}
\date{\empty}

\begin{document}
\textbf{Timetable Design Problem}
This problem has the following instance:
\begin{itemize}
	\item A set $H$ of work-periods.
	\item A set $W$ of workers.
	\item A set $T$ of tasks. 
	\item For each $w \in W$, a set $A(w) \ssteq H$ for when $w$ is able to work. 
	\item For each $t \in T$, a set $A(t)$ for when each task $t$ is available to be completed. 
	\item For each pair $(w,t) \in W\times T$ a number $R(w,t) \in \Z_0^+$ referring to number of times a worker $w$ has to work at task $t$. 
\end{itemize}

The corresponding decision problem is:

Is there a timetable for completing all tasks? The answer is in the form of a function $\func{f}{W\times T\times H}{\{0,1\}}$ where $f(w,t,h) = 1$ means that a worker $w$ works on task $t$ at time $h$. This function has to be subject ot the following constraints:

\begin{enumerate}
	\item $f(w,t,h) = 1$ only if $h \in A(w) \cap A(t)$. 
	\item  For each $h \in H$ and $w \in W$, there is at most one $t \in T$ for which $f(w,t,h) = 1$. 
	\item For each $h \in H$ and $t \in T$, there is at most $w \in W$ for which $f(w,t,h) = 1$.
	\item  For each pair $(w,t) \in W \times T$ there are exactly $R(w,t)$ values of $h$ for which $f(w,t,h) = 1$. 
\end{enumerate}


This problem is known to be NP-complete. You could get rid of requirement 3. I am not 100\% sure if this makes the problem easier or not. It might lead to trivial positive answers where every worker is working on 1 task. You might want have a number $M(t)$ for the maximum number of workers that can work on a task. I think with this additional criteria, we are back at the original problem since a task $t$ can be split into tasks $t_1, \ldots t_{M(t)}$. One thing this problem does not include is making sure every task is completed. 

\textbf{Basis Timetable Problem}
This problem has the following instance:
\begin{enumerate}
	\item A set $H$ of work-periods.
	\item A set $W$ of workers.
	\item A set $T$ of tasks.  
	\item For each $w \in W$, a set $A(w) \ssteq H$ for when $w$ is able to work. 
	\item For each $t \in T$, a set $A(t)$ for when each task $t$ is available to be completed. 
	\item A number $L(w) \in \Z_0^+$ for each $w \in W$ for the maximum number of tasks $w$ can work on. 
	\item A number $S(t) \in \Z_0^+$ for each $t \in T$ which is the maximum number of times to perform a task (i.e. different workers on the same task. These tasks are allowed to happen at different times unlike the comment for the previous problem I made.)
	\item A function $\func{A}{W \times T}{\{\text{true, false}\}}$ that determines if a worker can perform a task. The function is $A$ for adept.
\end{enumerate}
The corresponding decision problem is:

Is there a timetable for completing all the tasks? Is there a function $\func{f}{W\times T\times H}{\{0,1\}}$ as defined above that satisfies the following constraints. 

\begin{enumerate}
	\item $f(w,t,h) = 1$ only if $h \in A(w) \cap A(t)$.  
	\item  For each $h \in H$ and $w \in W$, there is at most one $t \in T$ for which $f(w,t,h) = 1$. 
	\item For each $w \in W, t \in T$ and $h \in H f(w,t,h) = 1$ only if $A(w,t) = $ true.
	\item For each $w \in W, \sum_{t,h} f(w,t,h) \le L(w)$.
	\item For each $t \in T, \sum_{w,h} f(w,t,h) \le S(t)$. 
\end{enumerate}

This problem is in P so there is a polynomial time algorithm that finds a solution to this problem. I believe it is constructive, I will have to read more. 

\textbf{Extended Timetable Problem}
This problem has the following instance:
\begin{itemize}
	\item A set $H$ of work-periods.
	\item A set $W$ of workers.
	\item A set $T$ of tasks.  
	\item A set $R$ of resources.
	\item For each $w \in W$, a set $A(w) \ssteq H$ for when $w$ is able to work.
	\item For each $t \in T$, a set $A(t)$ for when each task $t$ is available to be completed. 
	\item For each $r \in R$, a set $A(r) \ssteq H$ for when $r$ is available.
	\item A number $L(w) \in \Z_0^+$ for each $w \in W$ for the maximum number of tasks $w$ can work on. 
	\item A number $S(t) \in \Z_0^+$ for each $t \in T$ which is the maximum number of times to perform a task
	\item A number $U(r) \in \Z_0^+$ for each $r \in R$ which is the maximum number of tasks a resource can complete. (i.e., a supercomputer may only be able to run 1 job at a time)
	\item A function $\func{A}{W \times T}{\{\text{true, false}\}}$ that determines if a worker can perform a task. The function is $A$ for adept.
	\item A function $\func{RS}{T\times R}{\{\text{true, false}\}}$ for resource suitability where $RS(t,r)$ is true if a resource $r$ can be used for task $t$. 
\end{itemize}

This problem has the associated question:

Is there a timetable for completing all the tasks? Is there a function $\func{f}{W\times T\times H \times R}{\{0,1\}}$ where $f(w,t,h,r) = 1$ meaning worker $w$ completes task $t$ at time $h$ with resource $r$subject to the following constraints. 
\begin{enumerate}
	\item $f(w,t,h,r) = 1$ only if $h \in A(w) \cap A(t) \cap A(r)$. 
	\item For each pair $(w,h) \in W \times H$, there is at most one pair $(t,r) \in T \times R$ for which $f(w,t,h,r) = 1$. 
	\item For each pair $(r,h) \in R\times H$ there is at most one pair $(w,t) \in W \times T$ for which $f(w,t,h,r) = 1$. 
	\item For each $w \in W, t \in T, r \in R$ and $h \in H, f(w,t,h,r) = 1$ only if $A(w,t) =$ true.
	\item For each $w \in W, t \in T, r \in R$ and $h \in H, f(w,t,h,r) = 1$ only if $RS(t,r) = $ true.
	\item For each $w \in W, \sum_{t,h,r} f(w,t,h,r) \le L(w)$.
	\item For each $t \in T, \sum_{w,h,r} f(w,t,h,r) \le S(t)$.  
	\item For each $r \in R, \sum_{w,t,h} f(w,t,h,r) \le U(t)$.
\end{enumerate}
This problem is NP-complete.

These definitions came from \cite{lovelace2010complexity}. 
\bibliographystyle{plain}
\bibliography{biblio}
\end{document}
